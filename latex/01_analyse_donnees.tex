\chapter{Analyse des données}

\section{Visualisation des données}

Requête utile pour voir la répartition des causes de décés selon le sexe :

\begin{lstlisting}[frame=single]
    SELECT sexage.SEX, causes.LABEL, COUNT(*) AS Number
    FROM london INNER JOIN
            sexage ON london.SEXAGE = sexage.SEXAGE INNER JOIN
            causes on london.CAUSE = causes.CAUSE
    GROUP BY sexage.sex, causes.label
\end{lstlisting}

\section{Modification de données}

    Vu le format des données de la table SexAge, dissocier l'effet du sexe et de l'âge s'avérait hardu.

    Nous avons donc modifié la table en créant deux nouvelles colonnes: Sex, nvarchar qui contiendra le sexe (M/F), et Age,
    nvarchar qui contiendra la tranche d'âge (<1, 10-14, ...).

    Nous avons alors lancé la requête suivante afin de séparer les données de "sexage" sur ces deux colonnes :

    \begin{lstlisting}[frame=single]
        UPDATE sexage
        SET Sex=LEFT(label, 1), Age=RIGHT(label, LEN(label) - 1);
    \end{lstlisting}