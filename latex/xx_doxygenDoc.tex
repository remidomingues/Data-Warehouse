\chapter{Documentation sommaire doxygen}

\section{Brefs points essentiels}

Globalement, un commentaire se fera dans ce style la, avant une methode ou en tete de fichier:
\begin{lstlisting}
/**
 * ... text ...
 */
 \end{lstlisting}

Si vous voulez faire apparaitre un point important et resumant d abord, pour ensuite mettre des details il faut mettre :
\begin{lstlisting} 
/*! \brief Brief description.
 *         Brief description continued.
 *
 *  Detailed description starts here.
 */
 \end{lstlisting}

Des commentaires peuvent etre ajoutes dans les fonctions, par exemple apres certains attributs. Il faut alors placer un \"<\" :
\begin{lstlisting}
int var; /**< Detailed description after the member */
\end{lstlisting}

\phantomsection\label{balises}Lorsque vous documentez une enum, classe... il faut utiliser toujours le commentaire de base avec \\brief, mais en ajoutant au debut :
 \begin{lstlisting}
/**
    \struct to document a C-struct.
    \union to document a union.
    \enum to document an enumeration type.
    \fn to document a function.
    \var to document a variable or typedef or enum value.
    \def to document a #define.
    \typedef to document a type definition.
    \file to document a file.
    \namespace to document a namespace.
    \package to document a Java package.
    \interface to document an IDL interface.
    \bug to signal a bug
*/
 \end{lstlisting}


\section{Templates c++}

Tout d'abord, nous obtenons un template de fichier de cette forme :
\begin{lstlisting}
/**
 * \file main.c  ( specifie que vous desirez une doc sur ce fichier)
 * \brief Programme de tests.  (description breve du fichier)
 * \author Franck.H
 * \version 0.1
 * \date 11 septembre 2007
 *
 * Programme de test pour l'objet de gestion des chaines de caracteres Str_t.
 *
 */\end{lstlisting}
A placer au tout debut, il permet d'initier une documentations Doxygen sur le fichier, notamment grace a la balise \\file, a ne pas ommettre .





 Au final, on obtient un template de methode de ce type, avec le commentaire des parametres (entree ou sortie): 
 \begin{lstlisting}
 /**
 * \brief       Calcule la distance entre deux points
 * \details    La distance entre les \a point1 et \a point2 est calculee par l'intermediaire
 *                  des coordonnees des points. (cf #Point)
 * \param[in]   nom_param1         nom_param 1 pour le calcul de distance.
 * \param[out]    nom_param2         nom_param 2 retourner une valeur.
 * \return    Un \e float representant la distance calculee.
 */
\end{lstlisting}


\section{Exemples}

En utilisant les \hyperref[balises]{balises} decrites au dessus, nous pouvons deduire du commentaire :

\subsection{Un enum}

 \begin{lstlisting}
/**
 * \enum Str_err_e
 * \brief Constantes d'erreurs.
 *
 * Str_err_e est une serie de constantes predefinie pour diverses futures 
 * fonctions de l'objet Str_t.
 */
typedef enum
{
   STR_NO_ERR,    /*!< Pas d'erreur. */
   STR_EMPTY_ERR, /*!< Erreur: Objet vide ou non initialise. */
 
   NB_STR_ERR     /*!< Nombre total de constantes d'erreur. */
}
Str_err_e;
\end{lstlisting}

\subsection{Une struct} 
\begin{lstlisting}
/**
 * \struct Str_t
 * \brief Objet chaine de caracteres.
 *
 * Str_t est un petit objet de gestion de chaines de caracteres. 
 * La chaine se termine obligatoirement par un zero de fin et l'objet 
 * connait la taille de chaine contient !
 */
typedef struct
{
   char * sz;  /*!< Chaine avec  caractere null de fin de chaine. */
   size_t len; /*!< Taille de la chaine sz sans compter le zero de fin. */
}
Str_t;
\end{lstlisting}

Il est possible de signaler un bug : 
\begin{lstlisting}
/**
\bug { bug description }
*/ 
\end{lstlisting}
 

\subsection{Une fonction statique}

\begin{lstlisting}
 
/**
 * \fn static Str_t * str_new (const char * sz)
 * \brief Fonction de creation d'une nouvelle instance d'un objet Str_t.
 *
 * \param sz Chaine a stocker dans l'objet Str_t, ne peut etre NULL.
 * \return Instance nouvelle allouee d'un objet de type Str_t ou NULL.
 */
static Str_t * str_new (const char * sz)
{
   Str_t * self = NULL;
 
   if (sz != NULL && strlen (sz) > 0)
   {
      self = malloc (sizeof (* self));
 
      if (self != NULL)
      {
         self->len = strlen (sz);
         self->sz = malloc (self->len + 1);
 
         if (self->sz != NULL)
         {
            strcpy (self->sz, sz);
         }
         else
         {
            str_destroy (& self);
         }
      }
   }
 
   return self;
}
\end{lstlisting}


\subsection{exemple c++ complet}

\begin{lstlisting}

#ifndef CPLAYER_H_
#define CPLAYER_H_
 
/*!
 * \file CPlayer.h
 * \brief Lecteur de musique de base
 * \author hiko-seijuro
 * \version 0.1
 */
#include <string>
#include <list>
 
/*! \namespace player
 * 
 * espace de nommage regroupant les outils composants 
 * un lecteur audio
 */
namespace player
{
  /*! \class CPlayer
   * \brief classe representant le lecteur
   *
   *  La classe gere la lecture d'une liste de morceaux
   */
  class CPlayer
  {
  private:
    std::list<string> m_listSongs; /*!< Liste des morceaux*/
    std::list<string>::iterator m_currentSong; /*!< Morceau courant */
 
 
  public:
    /*!
     *  \brief Constructeur
     *
     *  Constructeur de la classe CPlayer
     *
     *  \param listSongs : liste initial des morceaux
     */
    CPlayer(std::list<string> listSongs);
 
    /*!
     *  \brief Destructeur
     *
     *  Destructeur de la classe CPlayer
     */
    virtual ~CPlayer();
 
  public:
    /*!
     *  \brief Ajout d'un morceau
     *
     *  Methode qui permet d'ajouter un morceau a liste de
     *  lecture
     *
     *  \param strSong : le morceau a ajouter
     *  \return true si morceau deja present dans la liste,
     *  false sinon
     */
    bool add(std::string strSong);
 
    /*!
     *  \brief Morceau suivant
     *
     *  Passage au morceau suivant
     */
    void next();
 
    /*!
     *  \brief Morceau precedent
     *
     *  Passage au morceau precedent
     */
    void previous();
 
    /*!
     *  \brief Lecture 
     *
     *  Lance la lecture de la liste
     */
    void play();
 
    /*!
     *  \brief Arret
     *
     *  Arrete la lecture
     */
    void stop();
  };
};
 
#endif
\end{lstlisting}
